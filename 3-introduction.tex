\introduction
%\paragraph{}bla bla bla \gls{acro} puis \Gls{acroglo} et enfin \gls{glossaire}

%1-	Problématique
\paragraph{}Les temps ont changé, l’heure est à la mobilité. Or les usages et les règles de sécurité qui fonctionnaient à domicile sur un \acrlong{pc} sont plus difficilement applicables en situation de mobilité, où l’usage du mot de passe atteint ses limites. Et pourtant, plus que jamais, il est primordial de protéger nos contenus, nos transactions et notre identité.
\paragraph{}La problématique de l'authentification par mot de passe ou code pin est liée à la sécurité des données et à la facilité d'utilisation. En effet, d'un côté, l'utilisation d'un mot de passe ou d'un code pin peut garantir la confidentialité des informations personnelles et des données sensibles, en empêchant l'accès aux tiers non autorisés. D'un autre côté, ces méthodes d'authentification peuvent être facilement compromises, par exemple, en utilisant des attaques de force brute ou de phishing, ou en découvrant le mot de passe ou le code pin par l'observation ou la divulgation.
\paragraph{}De plus, la complexité et la longueur des mots de passe peuvent rendre leur utilisation difficile et fastidieuse pour les utilisateurs, tandis que les codes pin peuvent être plus faciles à retenir mais également plus faciles à deviner.
\paragraph{}Ainsi, la question de l'authentification par mot de passe ou code pin est un compromis entre la sécurité et la convivialité, et doit être abordée avec prudence et en prenant en compte les risques potentiels et les besoins des utilisateurs. 
\paragraph{}La reconnaissance par biométrie vocale est une technologie qui utilise la voix comme moyen d'authentification et de reconnaissance d'identité. Elle peut être utilisée dans de nombreux domaines tels que la sécurité, l'accès à des services en ligne, la surveillance ou encore la gestion des appels téléphoniques. Cependant, cette technologie soulève des problématiques importantes en termes de protection de la vie privée, de fiabilité des systèmes et d'éthique. De plus, les données vocales collectées peuvent être utilisées à des fins malveillantes si elles tombent entre de mauvaises mains. Enfin, la biométrie vocale pose des questions quant à la protection de la vie privée et des libertés individuelles, notamment en ce qui concerne l'utilisation des données vocales pour la surveillance de masse ou la reconnaissance automatique de la voix dans les médias.
\paragraph{}Comment garantir la sécurité et la confidentialité des données collectées ? Comment éviter les erreurs de reconnaissance et les usurpations d'identité ? Quelles sont les conséquences éthiques de l'utilisation de la biométrie vocale ? Autant de questions auxquelles il est nécessaire de répondre pour assurer une utilisation responsable et efficace de cette technologie. 


%2-	Contexte et justif
\paragraph{}Face à la fraude documentaire et au vol d'identité, aux menaces du terrorisme ou de la cybercriminalité, et face à l'évolution logique des réglementations internationales, de nouvelles solutions technologiques sont progressivement mises en œuvre.
\paragraph{}Parmi ces technologies, la biométrie s'est rapidement distinguée comme la plus pertinente pour identifier et authentifier les personnes de manière fiable et rapide, en fonction de caractéristiques biologiques uniques.
\paragraph{}Aujourd'hui, de nombreuses applications font appel à cette technologie.
\paragraph{}Ce qui était autrefois réservé à des applications sensibles telles que la sécurisation de sites militaires est devenue une application grand public en développement rapide.
\paragraph{}Un simple geste de la main, une pression du doigt sur un capteur, une expression prononcée de la voix, ou un regard d’une seconde vers une caméra suffisent pour prouver son identité.
\paragraph{}L’authentification biométrique vocale facilite la vie des consommateurs et citoyens qui sont de plus en plus mobiles et connectés en leur apportant une alternative simple aux mots de passe et aux codes PIN pour s’authentifier. 
\paragraph{}Ainsi nous souhaiterons mettre en place un Cloud API pour l’authentification et l’identification des individus par biométrie vocale.

%3-	Objectifs
\paragraph{}L’objectif principal de notre étude est de concevoir déveloper et déployer un Cloud AI pouvant être appelé par REST API capable d’authentifier et d’identifier les individus à travers leur biometrie vocale. La matérialisation de cet objectif, passera par l’aboutissement des objectifs spécifiques que sont: 

\paragraph{}De façon spécifique, il s’agira :
\begin{itemize}
    \item la mise  en place la sauvegarde et l’organisation du dataset; 
    \item le choix des outils, des technologies  et de architecture à adopter ; 
    \item la mise en place de modèles d'IA  permettant les operations d’identification et d’authentification par biometrie vocale ; 
    \item la création mise en place des REST APIs;
\end{itemize}

%4-	 Organisation du mémoire
\paragraph{}Outre l’introduction et la conclusion, le present memoire comprend quatre (04) chapitres.
\paragraph{}Le premier chapitre présente l’état de l’art de l’identification par biometrie vocale.  Dans le deuxième chapitre nous explorons les materiels et méthodes. Le troisième chapitre nous permet d’exposer la modelisation et le developpement de notre model d’IA. Le quatrième chapitre nous permet de présenter les resultats découlant de nos travaux. Nous faisons ensuite une discussion et présentons les difficultés rencontreées dans la réalisation de  cette étude. 
