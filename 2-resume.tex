\resume
\selectlanguage{french}
\vspace*{-6cm}
\begin{abstract}
    \paragraph{}La parole est le moyen principal et le plus pratique de communication entre les gens. Que ce soit par curiosité technologique de construire des machines qui imitent les humains ou par le désir d'automatiser le travail avec des machines, la recherche en reconnaissance de la parole et des locuteurs, comme première étape vers une communication naturelle homme-machine, a suscité beaucoup d'enthousiasme au cours des cinq dernières décennies. 
    \paragraph{}La reconnaissance de locuteur (RL) est le processus d'identification d'un locuteur en fonction des caractéristiques vocales de la parole donnée. Cela diffère de la reconnaissance de la parole où le processus d'identification est limité au contenu plutôt qu'au locuteur. Le processus de SR repose sur l'identification et l'extraction des caractéristiques uniques de la parole du locuteur. Les caractéristiques de la voix de la personne sont également connues sous le nom de biométrie vocale.
    \paragraph{}Un système de Reconnaissance de Locuteur (RL) est utilisé pour identifier et distinguer les locuteurs et extraire des caractéristiques uniques qui peuvent être utilisées pour la vérification ou l'authentification de l'utilisateur. L'identification de locuteur (IL) est connue comme le processus d'identification du locuteur à partir d'une énonciation donnée en comparant la biométrie vocale de l'échantillon donné du locuteur. Lorsque la voix est utilisée pour l'autorisation, on parle de vérification du locuteur. Le domaine d'application clé de la RL est la sécurité et les sciences médico-légales. 
    \paragraph{}Les systèmes de RL sont également utilisés comme remplacement des mots de passe et d'autres processus d'authentification des utilisateurs (mot de passe vocal). Les sciences médico-légales appliquent la RL pour comparer les échantillons vocaux de la personne prétendument impliquée avec d'autres preuves obtenues, telles que des conversations téléphoniques ou d'autres preuves enregistrées. Ce processus est également appelé détection de locuteur. 
    \paragraph{}L'aspect le plus important de l'utilisation des systèmes d'IL est l'automatisation de processus tels que la redirection des courriers des clients vers la bonne boîte aux lettres, la reconnaissance des interlocuteurs dans une discussion, l'avertissement des systèmes de reconnaissance de discours en cas de changements de locuteur, la vérification si un client est enregistré dans le système, etc. Ces systèmes d'IL peuvent fonctionner sans la connaissance de l'échantillon vocal d'un client car ils se fient uniquement à l'identification d'un locuteur d'entrée à partir de la base de données existante de locuteurs.
    \paragraph{}Par conséquent, l'objectif de cette étude est de réaliser une revue systématique de la littérature sur les diverses approches de reconnaissance de locuteur et d’implémenter l’approche la plus prometteuse.

\paragraph{}
\textbf{Mots clés}: Reconnaissance de la parole et des locuteurs, biométrie vocale, échantillon vocal, biométrie vocale, reconnaissance de la parole
    
\end{abstract}

\newpage
\thispagestyle{empty}
\selectlanguage{english}
\addcontentsline{toc}{chapter}{Abstract}
\begin{abstract}

\paragraph{}Speech is the primary, and the most convenient means of communication between people. Whether due to technological curiosity to build machines that mimic humans or desire to automate work with machines, research in speech and speaker recognition, as a first step toward natural humanmachine communication, has attracted much enthusiasm over the past five decades.
\paragraph{}Speaker Recognition (SR) is the process of identifying the speaker according to the vocal features of the given speech. This is different to speech recognition where the identification process is confined to the content rather than speaker. The process of SR is based on identifying and extracting unique characteristics of the speaker's speech. The characteristics of voices of the person is also known as voice biometrics.
\paragraph{}A SR system is used to identify and distinguish speakers and extract unique characteristics that may be used for user verification or authentication. Speaker Identification (SI) is known as the process of identifying the speaker from a given utterance by comparing voice biometrics of the given sample of the speaker. 
\paragraph{}When voice is used for authorization, it is termed as Speaker Verification. The key application area of SR is security and forensic science. SR systems are also used as a replacement for password and other user authentication processes (voiced password). Forensic science applies SR to compare the voice samples of the person claimed to be with other evidences obtained like telephone conversation or other recorded evidence. 
\paragraph{}This process is also referred as speaker detection. The most important aspect of using SI systems is for automating processes like directing clients’ mails to the right mailbox, recognizing talkers in discussion, cautioning discourse acknowledgment frameworks of speaker changes, checking if a client is enlisted in the framework as of, and so on. These SI systems may work without the knowledge of a client's voice sample since they rely only on identifying an input speaker from the existing database of speakers.
\paragraph{}Therefore, the objective of this study is to conduct a systematic literature review on various speaker recognition approaches and implement the most promising approach.


\paragraph{}
\textbf{Key words}: research in speech and speaker recognition, Speaker Recognition (SR), speech recognition, voice biometrics, voice sample
\end{abstract}