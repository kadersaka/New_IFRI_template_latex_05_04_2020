\conclusion
%Bla bla bla \cite{ehrig2006graph}
\paragraph{}La mise en place d'une IA pour la reconnaissance de locuteur est un domaine en constante évolution qui présente des opportunités importantes pour de nombreuses applications dans diverses industries. Les avancées dans les domaines de l'apprentissage automatique, du traitement du signal et de la reconnaissance de la parole ont permis de développer des systèmes de reconnaissance de locuteur de plus en plus précis et efficaces.	
\paragraph{}Dans le domaine de la sécurité, l'IA peut être utilisée pour identifier des individus spécifiques en fonction de leur voix, ce qui peut aider à identifier des criminels ou des suspects dans une enquête. Dans les centres d'appels, l'IA peut aider à identifier automatiquement les appelants enregistrés, ce qui peut aider à accélérer le processus d'identification et améliorer l'efficacité du centre d'appels.
\paragraph{}Dans le domaine de l'assistance aux personnes âgées ou atteintes de troubles cognitifs, l'IA peut aider à identifier les membres de la famille ou les amis d'un patient par leur voix, ce qui peut aider à améliorer leur qualité de vie et leur sécurité.
\paragraph{}Cependant, la mise en place de ces systèmes soulève également des questions éthiques et de confidentialité. Il est important de garantir que les données des utilisateurs soient traitées de manière éthique et sécurisée, conformément aux lois et réglementations applicables.
\paragraph{}Pour l'avenir, il est possible que les systèmes de reconnaissance de locuteur soient intégrés dans une gamme encore plus large de dispositifs et de technologies, ce qui permettrait une interaction plus intuitive entre les humains et les machines. Par exemple, les voitures autonomes pourraient utiliser la reconnaissance vocale pour identifier les conducteurs et les passagers, ou les appareils médicaux pourraient être configurés pour reconnaître la voix d'un patient afin de personnaliser les soins.
\paragraph{}En fin de compte, la mise en place d'une IA pour la reconnaissance de locuteur présente à la fois des opportunités et des défis, et il est important d'évaluer soigneusement les avantages et les risques potentiels avant de l'utiliser dans une application spécifique.
\paragraph{}En perspectives, nous comptons ajouter d’autres fonctionnalités, poursuivre nos travaux de recherche afin de perfectionner au mieux le model pour aboutir à une solution beaucoup plus optimale. Il s’agira aussi de le mettre en place avec d’autres méthodes et algorithme de Deep Learning pour comparer les performances. 
